\documentclass[a4paper,11pt]{article}

% --- Packages ---
\usepackage[utf8]{inputenc}
\usepackage[T1]{fontenc}
\usepackage[english]{babel}
\usepackage{amsmath, amsfonts, amssymb}
\usepackage{geometry}
\usepackage{parskip} % Better paragraph spacing

% --- Layout ---
\geometry{hmargin=2.5cm,vmargin=2.5cm}

% --- Title ---
\title{Step 2: The Change of Measure\\(Girsanov's Theorem)}
\author{}
\date{}

\begin{document}

\maketitle

\section{The Problem: Real World Dynamics ($\mathbb{P}$)}

In the real world (historical probability measure $\mathbb{P}$), the asset price dynamics depend on the investors' risk aversion. The Stochastic Differential Equation (SDE) is:
\begin{equation}
    dS_t = \mu S_t dt + \sigma S_t dW_t^{\mathbb{P}}
\end{equation}
The parameter $\mu$ (real drift) is unknown, subjective, and difficult to estimate.
To price a derivative objectively, we need a framework that does not depend on $\mu$. We must move to the \textbf{Risk-Neutral World ($\mathbb{Q}$)}.

\section{The Market Price of Risk}

First, we define the "Market Price of Risk" $\lambda$. It represents the excess return earned per unit of volatility:
\begin{equation}
    \lambda = \frac{\mu - r}{\sigma}
\end{equation}
Rearranging this term gives us a useful expression for $\mu$:
\begin{equation} \label{eq:mu}
    \mu = r + \sigma\lambda
\end{equation}

\section{Girsanov's Theorem}

Girsanov's Theorem allows us to change the probability measure from $\mathbb{P}$ to $\mathbb{Q}$. It states that we can define a new Brownian motion $W_t^{\mathbb{Q}}$ under the risk-neutral measure such that:
\begin{equation}
    dW_t^{\mathbb{Q}} = dW_t^{\mathbb{P}} + \lambda dt
\end{equation}
Intuitively, we are shifting the mean of the noise.
We can express the "real" Brownian motion in terms of the "risk-neutral" one:
\begin{equation} \label{eq:brownian}
    dW_t^{\mathbb{P}} = dW_t^{\mathbb{Q}} - \lambda dt
\end{equation}

\section{Derivation of Risk-Neutral Dynamics}

We now substitute equation (\ref{eq:brownian}) into the original asset dynamics equation:

\begin{align*}
    dS_t &= \mu S_t dt + \sigma S_t dW_t^{\mathbb{P}} \\
    dS_t &= \mu S_t dt + \sigma S_t \left( dW_t^{\mathbb{Q}} - \lambda dt \right)
\end{align*}

We expand the terms:
\begin{align*}
    dS_t &= \mu S_t dt + \sigma S_t dW_t^{\mathbb{Q}} - \sigma \lambda S_t dt \\
    dS_t &= (\mu - \sigma \lambda) S_t dt + \sigma S_t dW_t^{\mathbb{Q}}
\end{align*}

Using equation (\ref{eq:mu}) ($\mu = r + \sigma\lambda$), we can substitute $\mu - \sigma \lambda$ with $r$:
$$ \mu - \sigma \lambda = r $$

\section{Conclusion}

The dynamics of the asset under the Risk-Neutral measure $\mathbb{Q}$ become:

\begin{equation}
    \boxed{dS_t = r S_t dt + \sigma S_t dW_t^{\mathbb{Q}}}
\end{equation}

\textbf{Key Interpretation:}
\begin{itemize}
    \item The unknown parameter $\mu$ has disappeared.
    \item It has been replaced by the risk-free rate $r$, which is observable.
    \item In the world $\mathbb{Q}$, on average, the asset grows at the rate of a bank account. This justifies the use of $r$ in the Black-Scholes pricing formula.
\end{itemize}

\end{document}