\documentclass[a4paper,11pt]{article}

% --- Packages ---
\usepackage[utf8]{inputenc}
\usepackage[T1]{fontenc}
\usepackage[english]{babel}
\usepackage{amsmath, amsfonts, amssymb}
\usepackage{geometry}
\usepackage{parskip}
\usepackage{booktabs} % For nicer tables

% --- Layout ---
\geometry{hmargin=2.5cm,vmargin=2.5cm}

% --- Title ---
\title{Step 5: Put-Call Parity\\(No-Arbitrage Relation)}
\author{}
\date{}

\begin{document}

\maketitle

\section{Introduction}

Put-Call Parity is a fundamental principle that defines the relationship between the price of European Call ($C$) and Put ($P$) options with the same strike ($K$) and maturity ($T$).
It is derived using a static replication argument, independent of any model (it works for Black-Scholes, but also for any other model, as long as there is no arbitrage).

\section{Portfolio Construction}

We construct two portfolios, A and B, today (at time $t=0$).

\subsection*{Portfolio A: Fiduciary Call}
\begin{itemize}
    \item Long one Call option ($C$).
    \item Long a zero-coupon bond paying $K$ at maturity (Value today: $K e^{-rT}$).
\end{itemize}
\begin{equation}
    \Pi_A = C + K e^{-rT}
\end{equation}

\subsection*{Portfolio B: Protective Put}
\begin{itemize}
    \item Long one Put option ($P$).
    \item Long one share of the underlying stock ($S$).
\end{itemize}
\begin{equation}
    \Pi_B = P + S
\end{equation}

\section{Payoff Analysis at Maturity ($T$)}

We analyze the value of both portfolios at time $T$, depending on the final stock price $S_T$.

\begin{table}[h!]
    \centering
    \renewcommand{\arraystretch}{1.5}
    \begin{tabular}{lcc}
        \toprule
        \textbf{State of Market} & \textbf{$S_T \le K$} & \textbf{$S_T > K$} \\
        \midrule
        \textbf{Portfolio A} & & \\
        \quad Call Option $\max(S_T-K, 0)$ & $0$ & $S_T - K$ \\
        \quad Cash (Bond) & $K$ & $K$ \\
        \textbf{Total Payoff A} & $\mathbf{K}$ & $\mathbf{S_T}$ \\
        \midrule
        \textbf{Portfolio B} & & \\
        \quad Put Option $\max(K-S_T, 0)$ & $K - S_T$ & $0$ \\
        \quad Stock & $S_T$ & $S_T$ \\
        \textbf{Total Payoff B} & $\mathbf{K}$ & $\mathbf{S_T}$ \\
        \bottomrule
    \end{tabular}
    \caption{Payoff Matrix}
\end{table}

\section{Conclusion}

As demonstrated in the table, both portfolios deliver exactly the same payoff in all possible states of the world:
$$ \Pi_A(T) = \max(S_T, K) = \Pi_B(T) $$

By the \textbf{Law of One Price} (No-Arbitrage Principle), if two assets have identical future cash flows, they must have the same price today.

Therefore:
\begin{equation}
    \Pi_A(0) = \Pi_B(0)
\end{equation}

This yields the fundamental Put-Call Parity equation:

\begin{equation}
    \boxed{C + K e^{-rT} = P + S}
\end{equation}

\textbf{Application:}
If we know the price of the Call (from Black-Scholes), we can immediately find the price of the Put without evaluating a new integral:
$$ P = C - S + K e^{-rT} $$

\end{document}