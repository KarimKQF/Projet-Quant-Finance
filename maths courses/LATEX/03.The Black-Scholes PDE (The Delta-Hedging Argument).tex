\documentclass[a4paper,11pt]{article}

% --- Packages ---
\usepackage[utf8]{inputenc}
\usepackage[T1]{fontenc}
\usepackage[english]{babel}
\usepackage{amsmath, amsfonts, amssymb}
\usepackage{geometry}
\usepackage{parskip}

% --- Layout ---
\geometry{hmargin=2.5cm,vmargin=2.5cm}

% --- Title ---
\title{Step 3: The Black-Scholes PDE\\(The Delta-Hedging Argument)}
\author{}
\date{}

\begin{document}

\maketitle

\section{Introduction: The Hedging Argument}

Instead of using probabilistic expectations, we can derive the option price by constructing a risk-free portfolio.
This approach relies on the assumption of \textbf{No-Arbitrage}: If a portfolio has no risk, it must earn the risk-free rate $r$.

\section{Constructing the Portfolio}

Consider a portfolio $\Pi$ consisting of:
\begin{itemize}
    \item A long position in one option $V(S, t)$.
    \item A short position in $\Delta$ units of the underlying asset $S$.
\end{itemize}

The value of the portfolio is:
\begin{equation}
    \Pi = V(S, t) - \Delta S
\end{equation}

The change in the value of the portfolio over a small time interval $dt$ is:
\begin{equation} \label{eq:dpi}
    d\Pi = dV - \Delta dS
\end{equation}

\section{Applying Ito's Lemma}

Since $V$ is a function of $S$ and $t$, we apply Ito's Lemma to find $dV$:
\begin{equation}
    dV = \frac{\partial V}{\partial t}dt + \frac{\partial V}{\partial S}dS + \frac{1}{2}\sigma^2 S^2 \frac{\partial^2 V}{\partial S^2}dt
\end{equation}

Substitute this into equation (\ref{eq:dpi}):
$$ d\Pi = \left( \frac{\partial V}{\partial t}dt + \frac{\partial V}{\partial S}dS + \frac{1}{2}\sigma^2 S^2 \frac{\partial^2 V}{\partial S^2}dt \right) - \Delta dS $$

Rearranging terms to group $dt$ and $dS$:
\begin{equation}
    d\Pi = \left( \frac{\partial V}{\partial t} + \frac{1}{2}\sigma^2 S^2 \frac{\partial^2 V}{\partial S^2} \right)dt + \left( \frac{\partial V}{\partial S} - \Delta \right)dS
\end{equation}

\section{Eliminating Risk (Delta Hedging)}

The term $dS$ contains the stochastic component ($dW_t$). To make the portfolio risk-free, we must eliminate this source of uncertainty. We choose $\Delta$ such that the coefficient of $dS$ is zero:

\begin{equation}
    \frac{\partial V}{\partial S} - \Delta = 0 \quad \implies \quad \Delta = \frac{\partial V}{\partial S}
\end{equation}

This is the mathematical definition of the \textbf{Delta} of the option. The portfolio dynamics are now deterministic (no randomness):
\begin{equation} \label{eq:riskfree1}
    d\Pi = \left( \frac{\partial V}{\partial t} + \frac{1}{2}\sigma^2 S^2 \frac{\partial^2 V}{\partial S^2} \right)dt
\end{equation}

\section{No-Arbitrage Condition}

Since the portfolio $\Pi$ is risk-free, it must grow at the risk-free rate $r$ to prevent arbitrage opportunities:
\begin{equation}
    d\Pi = r \Pi dt
\end{equation}

Substituting $\Pi = V - \Delta S$:
\begin{equation} \label{eq:riskfree2}
    d\Pi = r \left( V - \frac{\partial V}{\partial S} S \right) dt
\end{equation}

\section{The Black-Scholes Equation}

We equate (\ref{eq:riskfree1}) and (\ref{eq:riskfree2}):
$$ \left( \frac{\partial V}{\partial t} + \frac{1}{2}\sigma^2 S^2 \frac{\partial^2 V}{\partial S^2} \right)dt = r \left( V - S \frac{\partial V}{\partial S} \right) dt $$

Dividing by $dt$ and rearranging all terms to one side, we obtain the Black-Scholes Partial Differential Equation (PDE):

\begin{equation}
    \boxed{\frac{\partial V}{\partial t} + rS \frac{\partial V}{\partial S} + \frac{1}{2}\sigma^2 S^2 \frac{\partial^2 V}{\partial S^2} - rV = 0}
\end{equation}

\textbf{Note:} This is a heat equation (diffusion equation). Solving this PDE with the boundary condition $\max(S_T - K, 0)$ yields the exact same Black-Scholes formula derived in Step 1 and 2.

\end{document}