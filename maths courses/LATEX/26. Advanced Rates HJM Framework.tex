\documentclass[a4paper,11pt]{article}

% --- Packages ---
\usepackage[utf8]{inputenc}
\usepackage[T1]{fontenc}
\usepackage[english]{babel}
\usepackage{amsmath, amsfonts, amssymb}
\usepackage{geometry}
\usepackage{parskip}
\usepackage{fancyhdr}

% --- Layout ---
\geometry{hmargin=2.5cm,vmargin=2.5cm}
\pagestyle{fancy}
\fancyhead[L]{Advanced Rates Modeling}
\fancyhead[R]{Quant Finance Portfolio}

\title{\textbf{Step 26: Advanced Rates Modeling}\\ \large The Heath-Jarrow-Morton (HJM) Framework}
\author{}
\date{}

\begin{document}

\maketitle
\tableofcontents

\section{The Philosophy of Forward Rates}
Short-rate models (Vasicek, CIR) start with $r_t$ and try to fit the yield curve. This often fails or requires time-dependent parameters.
\textbf{David Heath, Robert Jarrow, and Andrew Morton (1992)} inverted the problem: Why not take the observed current Forward Curve $f(0,T)$ as an input and model its evolution?

\section{Mathematical Formulation}
Let $f(t,T)$ be the instantaneous forward rate at time $t$ for maturity $T$.
$$ P(t,T) = \exp\left( -\int_t^T f(t,u) du \right) $$

We assume that for every maturity $T$, $f(t,T)$ follows an Itô process:
\begin{equation}
    df(t,T) = \alpha(t,T) dt + \sigma(t,T) dW_t
\end{equation}

\section{The HJM No-Arbitrage Drift Condition}
This is the central theorem. If the market is free of arbitrage, we cannot choose the drift $\alpha$ and volatility $\sigma$ independently.

Under the Risk-Neutral Measure $\mathbb{Q}$, the discounted bond price $Z(t,T) = D_t P(t,T)$ must be a martingale.
Applying Ito's Lemma to the bond price formula leads to the \textbf{HJM Drift Condition}:

\begin{equation}
    \boxed{\alpha(t,T) = \sigma(t,T) \int_t^T \sigma(t,u) du}
\end{equation}

\textbf{Interpretation:} The drift of the forward rate is completely determined by its volatility structure. High volatility implies high drift (convexity correction).

\section{Gaussian HJM and Principal Component Analysis}
If $\sigma(t,T)$ is deterministic (Gaussian HJM), the forward rates are normally distributed.
In practice, the curve does not move in 1 dimension. We typically use a multi-factor model:
$$ df(t,T) = \alpha(t,T) dt + \sum_{k=1}^3 \sigma_k(t,T) dW_t^{(k)} $$
PCA Analysis of yield curve changes reveals 3 factors:
\begin{enumerate}
    \item \textbf{Level} (Shift): 90\% of variance.
    \item \textbf{Slope} (Tilt): 8\% of variance.
    \item \textbf{Curvature} (Twist): 2\% of variance.
\end{enumerate}

\section{The Brace-Gatarek-Musiela (BGM) / LIBOR Market Model}
HJM models instantaneous rates $f(t,T)$ which are abstract. The LMM models **Forward LIBOR rates** $L(t, T_i, T_{i+1})$, which are the underlying assets of Caps and Swaptions.
$$ \frac{dL_k(t)}{L_k(t)} = \mu_k(t) dt + \sigma_k(t) dW_t $$
The LMM is essentially the "discrete, log-normal" version of HJM. It is the gold standard for pricing complex interest rate derivatives (e.g., Bermudan Swaptions, Snowballs).

\section{Monte Carlo Implementation}
To simulate the HJM framework:
\begin{enumerate}
    \item Discretize time into steps $\Delta t$.
    \item Initiate with the current Market Forward Curve $f(0, T)$.
    \item At each step, update the entire curve:
    $$ f(t+\Delta t, T) = f(t,T) + \left(\sigma \int_t^T \sigma \right) \Delta t + \sigma \sqrt{\Delta t} Z $$
    \item Calculate Bond Prices $P(t,T)$ from the integral of the new curve.
\end{enumerate}

\end{document}
