\documentclass[a4paper,11pt]{article}

% --- Packages ---
\usepackage[utf8]{inputenc}
\usepackage[T1]{fontenc}
\usepackage[english]{babel}
\usepackage{amsmath, amsfonts, amssymb}
\usepackage{geometry}
\usepackage{parskip}
\usepackage{graphicx}
\usepackage{fancyhdr}

% --- Layout ---
\geometry{hmargin=2.5cm,vmargin=2.5cm}

% --- Title ---
\title{Step 7: Implied Volatility\\(The Inverse Problem)}
\author{}
\date{}

\begin{document}

\maketitle

\section{Introduction: Price vs. Volatility}

In the previous steps, we treated volatility ($\sigma$) as an input parameter to finding the option price ($C$). This is the theoretical approach.
$$ \text{Input: } (S, K, r, T, \sigma) \xrightarrow{\text{Black-Scholes}} \text{Output: Price} $$

 However, in professional trading, the logic is reversed.
\begin{itemize}
    \item \textbf{Observable:} The Option Price ($C_{mkt}$) is determined by supply and demand on the exchange.
    \item \textbf{Unobservable:} Future volatility is unknown.
\end{itemize}

Therefore, traders use the Black-Scholes formula as a translation tool to convert a market price (in dollars/euros) into a volatility figure (in \%). This output is called **Implied Volatility** ($\sigma_{imp}$). It represents the market's expectation of the average volatility until maturity.

\section{Mathematical Formulation}

We seek the value $\sigma_{imp}$ that solves the following equation:
\begin{equation} \label{eq:inverse}
    C_{BS}(\sigma_{imp}) - C_{mkt} = 0
\end{equation}

Where $C_{BS}(\sigma)$ is the Black-Scholes call price function:
$$ C_{BS}(\sigma) = S_0 N(d_1(\sigma)) - K e^{-rT} N(d_2(\sigma)) $$

\subsection{The Problem of Inversion}
This equation is **non-linear and transcendental**. Because $\sigma$ appears inside the limits of the normal integral (via $d_1$ and $d_2$), it is algebraically impossible to isolate $\sigma$.
$$ \sigma = \text{Formula}(C_{mkt}) \quad \leftarrow \textbf{Impossible} $$

\subsection{Existence and Uniqueness (Bijectivity)}
Before trying to solve it numerically, we must ensure a solution exists.
Recall from Step 7 that **Vega** is the derivative of price with respect to volatility:
$$ \mathcal{V} = \frac{\partial C}{\partial \sigma} = S \sqrt{T} \varphi(d_1) $$

Since $S > 0$, $T > 0$, and the Gaussian density $\varphi(d_1) > 0$, we have:
$$ \mathcal{V} > 0 \quad \text{for all } \sigma > 0 $$

\textbf{Conclusion:} The pricing function is strictly monotonic (increasing).
\begin{itemize}
    \item If $\sigma \to 0$, $C \to \max(S-Ke^{-rT}, 0)$ (Intrinsic Value).
    \item If $\sigma \to \infty$, $C \to S$ (The option becomes the stock).
\end{itemize}
As long as the market price $C_{mkt}$ is strictly greater than the intrinsic value (which is always true for traded options due to time value), a unique solution $\sigma_{imp}$ exists.

\section{Numerical Resolution: Newton-Raphson}

To find the root of $f(\sigma) = C_{BS}(\sigma) - C_{mkt} = 0$, we use the Newton-Raphson algorithm. This method is extremely fast because we have an analytical formula for the derivative (Vega).

\subsection{Geometric Interpretation}
Imagine we are at a guess point $\sigma_n$. We approximate the complex Black-Scholes curve by its **tangent line** at that point.
The slope of this tangent is Vega ($\mathcal{V}(\sigma_n)$).
We follow this tangent line down to zero to find our next guess $\sigma_{n+1}$.

\subsection{The Algorithm}
The Taylor expansion around $\sigma_n$ gives:
$$ C_{BS}(\sigma_{imp}) \approx C_{BS}(\sigma_n) + \mathcal{V}(\sigma_n)(\sigma_{imp} - \sigma_n) $$
We set the left side to the target market price $C_{mkt}$:
$$ C_{mkt} \approx C_{BS}(\sigma_n) + \mathcal{V}(\sigma_n)(\sigma_{imp} - \sigma_n) $$

Solving for $\sigma_{imp}$ (which becomes our next step $\sigma_{n+1}$):

\begin{equation}
    \boxed{\sigma_{n+1} = \sigma_n - \frac{C_{BS}(\sigma_n) - C_{mkt}}{\mathcal{V}(\sigma_n)}}
\end{equation}

This iterative process typically converges to machine precision ($10^{-8}$) in 3 to 5 iterations.

\section{Market Reality: The Volatility Surface}

If the Black-Scholes assumptions were perfectly true (Log-Normal distribution, constant volatility), calculating $\sigma_{imp}$ for options with different strikes ($K$) and maturities ($T$) should yield the same number.

\textbf{Reality check:} This is not what we observe.
When plotting Implied Volatility against Strike Price $K$, we observe a curve called the **Volatility Smile** or **Skew**.

\begin{itemize}
    \item \textbf{The Smile (Forex):} Implied volatility is higher for both deep ITM and deep OTM options. This suggests the market fears extreme moves (fat tails) more than the Gaussian model predicts.
    \item \textbf{The Skew (Equities):} Low strike Puts have much higher volatility than high strike Calls. This is known as "Crash Phobia." Traders pay a premium for insurance against market drops, inflating the price (and thus the implied vol) of Puts.
\end{itemize}

This leads to the concept of the \textbf{Volatility Surface} $\Sigma(K, T)$, implying that Black-Scholes is used as a quoting convention rather than a perfect physical model.

\end{document}