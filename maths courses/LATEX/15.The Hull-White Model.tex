\documentclass[a4paper,11pt]{report}

% =======================================================
% PACKAGES & CONFIGURATION
% =======================================================
\usepackage[utf8]{inputenc}
\usepackage[T1]{fontenc}
\usepackage[english]{babel}
\usepackage{amsmath, amsfonts, amssymb, amsthm}
\usepackage{geometry}
\usepackage{parskip}
\usepackage{fancyhdr}
\usepackage{tcolorbox}
\usepackage{hyperref}
\usepackage{booktabs}

% --- Graphics & Plots ---
\usepackage{pgfplots}
\pgfplotsset{compat=1.18}
\usepackage{tikz}

% --- Layout Settings ---
\geometry{hmargin=2.5cm,vmargin=2.5cm}
\definecolor{hwGreen}{RGB}{0, 100, 50}
\definecolor{vasBlue}{RGB}{0, 60, 120}

% --- Header/Footer ---
\pagestyle{fancy}
\fancyhead[L]{\textbf{Fixed Income Modeling II}}
\fancyhead[R]{The Hull-White Model}
\fancyfoot[C]{\thepage}

% --- Theorem Styles ---
\newtheorem{definition}{Definition}[chapter]
\newtheorem{theorem}{Theorem}[chapter]
\newtheorem{remark}{Remark}[chapter]

% =======================================================
% DOCUMENT START
% =======================================================
\begin{document}

% --- TITLE PAGE ---
\begin{titlepage}
    \centering
    \vspace*{2cm}
    {\Huge \textbf{The Hull-White Model}}\\[0.5cm]
    {\Large \textit{Perfect Calibration to the Term Structure}}\\[2cm]
    
    \textbf{Lecture 15}\\
    M.Sc. Quantitative Finance\\[2cm]
    
    % Graph: Vasicek vs Hull-White
    \begin{tikzpicture}
        \begin{axis}[
            width=12cm, height=7cm,
            xlabel={Time ($t$)}, ylabel={Short Rate Expectation $\mathbb{E}[r_t]$},
            xmin=0, xmax=30, ymin=0, ymax=0.08,
            grid=major,
            title={\textbf{Vasicek vs. Hull-White Mean Reversion}}
        ]
        
        % Vasicek: Reverts to constant level
        \addplot[vasBlue, dashed, ultra thick, domain=0:30] {0.03 + (0.05-0.03)*(1-exp(-0.1*x))};
        \addlegendentry{Vasicek (Constant $\theta$)}
        
        % Hull-White: Fits the market forward curve (e.g. inverted then hump)
        \addplot[hwGreen, ultra thick, domain=0:30] {0.03 + 0.03*sin(10*x) + 0.001*x};
        \addlegendentry{Hull-White (Time-Dep $\theta(t)$)}
        
        \node at (axis cs: 15, 0.02) {HW follows the Market Curve};
        
        \end{axis}
    \end{tikzpicture}
    
    \vfill
    {\large \today}
\end{titlepage}

\tableofcontents
\newpage

% =======================================================
% CHAPTER 1: EXTENDING VASICEK
% =======================================================
\chapter{The Hull-White Model (1990)}

\section{The Fitting Problem}
The Vasicek model ($dr_t = a(b - r_t)dt + \sigma dW_t$) depends on 3 constant parameters: $a, b, \sigma$.
The market Yield Curve contains dozens of data points (1M, 3M, 6M, 1Y, ... 30Y).
It is mathematically impossible for 3 constants to fit 30 points perfectly.
\textbf{Consequence:} If you use Vasicek to price a bond, your model price $P^{Model}$ will differ from the market price $P^{Mkt}$. This allows for arbitrage, which is unacceptable.

\section{The Hull-White Extension}
Hull and White (1990) introduced a time-dependent parameter $\theta(t)$ to absorb the error at every maturity.

\begin{tcolorbox}[colback=green!5!white,colframe=hwGreen,title=\textbf{Hull-White SDE}]
\begin{equation}
    dr_t = (\theta(t) - a r_t)dt + \sigma dW_t
\end{equation}
\end{tcolorbox}
Here, $a$ (mean reversion speed) and $\sigma$ (volatility) are usually kept constant, but $\theta(t)$ changes over time.

\section{Fitting to the Forward Curve}
We want the model to match the initial term structure of interest rates observed in the market.
Let $f^M(0, t)$ be the market instantaneous forward rate at time 0 for maturity $t$.
$$ f^M(0, t) = - \frac{\partial \ln P^{Mkt}(0, t)}{\partial t} $$

Hull and White derived the exact functional form required for $\theta(t)$:

\begin{theorem}[Calibration Formula]
To fit the initial term structure perfectly, $\theta(t)$ must be:
\begin{equation} \label{eq:theta}
    \theta(t) = \frac{\partial f^M(0,t)}{\partial t} + a f^M(0,t) + \frac{\sigma^2}{2a}(1 - e^{-2at})
\end{equation}
\end{theorem}

\begin{remark}
The term $\frac{\sigma^2}{2a}(1 - e^{-2at})$ is a "convexity adjustment". It arises because bond prices are non-linear functions of rates (Jensen's Inequality).
\end{remark}

% =======================================================
% CHAPTER 2: PRICING FORMULAS
% =======================================================
\chapter{Analytical Pricing}

Since Hull-White is still a Gaussian model (just like Vasicek), it retains the beautiful Affine structure.

\section{Zero-Coupon Bonds}
The price of a Zero-Coupon Bond paying \$1 at $T$ is:
\begin{equation}
    P(t, T) = A(t, T) e^{-B(t, T) r_t}
\end{equation}

Where $B(t, T)$ is the same as in Vasicek:
$$ B(t, T) = \frac{1 - e^{-a(T-t)}}{a} $$

But $A(t, T)$ is adjusted to ensure the fit:
$$ \ln A(t, T) = \ln \frac{P^{Mkt}(0, T)}{P^{Mkt}(0, t)} - B(t, T) f^M(0, t) - \frac{\sigma^2}{4a} (1 - e^{-2at}) B(t, T)^2 $$

\section{Option on Bonds (Call on ZCB)}
A European Call option on a Zero-Coupon Bond (maturity $S$, strike $K$, option expiry $T < S$) has a Black-Scholes-like formula:

\begin{tcolorbox}[colback=blue!5!white,colframe=hwGreen,title=\textbf{ZCB Option Price}]
\begin{equation}
    C_t = P(t, S) N(h) - K P(t, T) N(h - \sigma_p)
\end{equation}
Where:
$$ \sigma_p = \frac{\sigma}{a} \left( 1 - e^{-a(S-T)} \right) \sqrt{\frac{1 - e^{-2a(T-t)}}{2a}} $$
$$ h = \frac{1}{\sigma_p} \ln \frac{P(t, S)}{P(t, T) K} + \frac{\sigma_p}{2} $$
\end{tcolorbox}

% =======================================================
% CHAPTER 3: APPLICATIONS
% =======================================================
\chapter{Swaptions \& Jamshidian's Decomposition}

\section{The Swaption Problem}
A \textbf{Swaption} gives the right to enter a Swap. A Swap can be viewed as a portfolio of Zero-Coupon Bonds (a coupon-bearing bond equals par).
So a Swaption is an option on a portfolio of bonds:
$$ \text{Payoff} = \max\left( \sum c_i P(T, T_i) - K, 0 \right) $$

\section{Jamshidian's Decomposition (1989)}
In general, an option on a sum is NOT the sum of options ($\max(\sum x) \neq \sum \max(x)$).
However, in 1-factor models like Hull-White, all bond prices $P(T, T_i)$ are driven by the same variable $r_T$. They move in lockstep (monotonicity).

We can find a critical rate $r^*$ such that the portfolio equals $K$.
Then the Swaption decomposes into a sum of options on individual Zero-Coupon Bonds, each with its own strike $K_i = P(r^*, T, T_i)$.

\begin{equation}
    \text{Swaption} = \sum_{i} c_i \cdot \text{Option}_{ZCB}(K_i)
\end{equation}

This allows for ultra-fast pricing of Swaptions using the formula from Chapter 2.

\end{document}