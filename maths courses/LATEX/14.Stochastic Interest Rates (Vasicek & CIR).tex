\documentclass[a4paper,11pt]{report}

% =======================================================
% PACKAGES & CONFIGURATION
% =======================================================
\usepackage[utf8]{inputenc}
\usepackage[T1]{fontenc}
\usepackage[english]{babel}
\usepackage{amsmath, amsfonts, amssymb, amsthm}
\usepackage{geometry}
\usepackage{parskip}
\usepackage{fancyhdr}
\usepackage{tcolorbox}
\usepackage{hyperref}
\usepackage{booktabs}

% --- Graphics & Plots ---
\usepackage{pgfplots}
\pgfplotsset{compat=1.18}
\usepackage{tikz}

% --- Layout Settings ---
\geometry{hmargin=2.5cm,vmargin=2.5cm}
\definecolor{rateBlue}{RGB}{0, 60, 120}
\definecolor{rateRed}{RGB}{180, 40, 40}

% --- Header/Footer ---
\pagestyle{fancy}
\fancyhead[L]{\textbf{Fixed Income Modeling}}
\fancyhead[R]{Vasicek \& CIR}
\fancyfoot[C]{\thepage}

% --- Theorem Styles ---
\newtheorem{definition}{Definition}[chapter]
\newtheorem{theorem}{Theorem}[chapter]

% =======================================================
% DOCUMENT START
% =======================================================
\begin{document}

% --- TITLE PAGE ---
\begin{titlepage}
    \centering
    \vspace*{2cm}
    {\Huge \textbf{Stochastic Interest Rates}}\\[0.5cm]
    {\Large \textit{Vasicek, CIR, and Bond Pricing}}\\[2cm]
    
    \textbf{Lecture 14}\\
    M.Sc. Quantitative Finance\\[2cm]
    
    % Mean Reversion Graph
    \begin{tikzpicture}
        \begin{axis}[
            width=12cm, height=7cm,
            xlabel={Time ($t$)}, ylabel={Short Rate ($r_t$)},
            xmin=0, xmax=50, ymin=0, ymax=0.10,
            grid=major,
            title={\textbf{Mean Reverting Rates}}
        ]
        
        % Target Mean Level (b)
        \addplot[black, dashed, ultra thick] coordinates {(0,0.05) (50,0.05)};
        \node at (axis cs: 45, 0.052) {Long Term Mean $b$};
        
        % Simulated Vasicek Path
        \addplot[rateBlue, thick] coordinates {
            (0,0.02) (5,0.035) (10,0.045) (15,0.055) (20,0.048) 
            (25,0.052) (30,0.042) (35,0.038) (40,0.045) (45,0.051) (50,0.049)
        };
        \addlegendentry{Rate Path $r_t$}
        
        \end{axis}
    \end{tikzpicture}
    
    \vfill
    {\large \today}
\end{titlepage}

\tableofcontents
\newpage

% =======================================================
% CHAPTER 1: MODELING THE SHORT RATE
% =======================================================
\chapter{The Short Rate Framework}

\section{Introduction}
So far, we assumed the risk-free rate $r$ was constant. In Fixed Income markets, $r$ is stochastic.
We model the instantaneous spot rate, or \textbf{Short Rate} $r_t$.
The price of a Zero-Coupon Bond paying \$1 at time $T$ is given by the risk-neutral expectation:

\begin{equation}
    P(t, T) = \mathbb{E}^{\mathbb{Q}} \left[ \exp\left( - \int_t^T r_u du \right) \mid \mathcal{F}_t \right]
\end{equation}

To calculate this, we need an SDE for $r_t$.

\section{The Vasicek Model (1977)}
Vasicek assumed $r_t$ follows an Ornstein-Uhlenbeck process (Mean-Reverting Gaussian).

\begin{tcolorbox}[colback=blue!5!white,colframe=rateBlue,title=\textbf{Vasicek SDE}]
\begin{equation}
    dr_t = a(b - r_t)dt + \sigma dW_t
\end{equation}
\end{tcolorbox}
\begin{itemize}
    \item $b$: Long-term mean level (e.g., 5\%).
    \item $a$: Speed of mean reversion.
    \item $\sigma$: Volatility of rates.
\end{itemize}

\textbf{Properties:}
\begin{itemize}
    \item \textbf{Gaussian:} $r_T$ given $r_t$ is normally distributed.
    \item \textbf{Negative Rates:} Since it is Gaussian, $r_t$ can become negative. (Historically considered a flaw, but realistic in post-2008 Europe/Japan).
\end{itemize}

\section{The Cox-Ingersoll-Ross (CIR) Model (1985)}
To prevent negative interest rates, CIR introduced a square-root diffusion term (similar to Heston).

\begin{tcolorbox}[colback=red!5!white,colframe=rateRed,title=\textbf{CIR SDE}]
\begin{equation}
    dr_t = a(b - r_t)dt + \sigma \sqrt{r_t} dW_t
\end{equation}
\end{tcolorbox}

\textbf{Properties:}
\begin{itemize}
    \item \textbf{Non-Central Chi-Squared:} The distribution of $r_t$ is no longer Gaussian.
    \item \textbf{Positivity:} If $2ab \ge \sigma^2$ (Feller Condition), rates stay strictly positive.
\end{itemize}

% =======================================================
% CHAPTER 2: BOND PRICING (AFFINE MODELS)
% =======================================================
\chapter{Analytical Bond Pricing}

Both Vasicek and CIR belong to the class of \textbf{Affine Term Structure Models}. This means the bond price has a specific exponential form.

\section{The Affine Formula}
For both models, the price of a Zero-Coupon Bond is:

\begin{equation} \label{eq:affine}
    P(t, T) = A(t, T) e^{-B(t, T) r_t}
\end{equation}

This is incredibly powerful: knowing the current rate $r_t$ is enough to generate the entire Yield Curve.

\section{Vasicek Solution}
Let $\tau = T - t$.
\begin{align}
    B(\tau) &= \frac{1 - e^{-a\tau}}{a} \\
    A(\tau) &= \exp \left( \left( b - \frac{\sigma^2}{2a^2} \right)(B(\tau) - \tau) - \frac{\sigma^2}{4a} B(\tau)^2 \right)
\end{align}

\section{The Yield Curve}
The Yield $Y(t,T)$ is defined as $-\frac{\ln P(t,T)}{\tau}$.
$$ Y(t,T) = \frac{B(\tau)}{\tau} r_t - \frac{\ln A(\tau)}{\tau} $$

\begin{center}
\begin{tikzpicture}
    \begin{axis}[
        width=10cm, height=6cm,
        xlabel={Maturity $T$ (Years)}, ylabel={Yield \%},
        title={Vasicek Yield Curves (Different $r_0$)},
        legend pos=south east, grid=major
    ]
    
    % Low initial rate -> Upward sloping curve
    \addplot[rateBlue, thick, domain=0:30] {0.05 - (0.05-0.01)*exp(-0.1*x)};
    \addlegendentry{$r_0 = 1\%$ (Contango)}
    
    % High initial rate -> Inverted curve
    \addplot[rateRed, thick, domain=0:30] {0.05 + (0.10-0.05)*exp(-0.1*x)};
    \addlegendentry{$r_0 = 10\%$ (Inverted)}
    
    % Mean level
    \addplot[black, dashed] coordinates {(0,0.05) (30,0.05)};
    
    \end{axis}
\end{tikzpicture}
\end{center}

\end{document}