\documentclass[a4paper,11pt]{article}
\usepackage[utf8]{inputenc}
\usepackage[T1]{fontenc}
\usepackage[english]{babel} % Changed to English
\usepackage{amsmath, amsfonts, amssymb}
\usepackage{geometry}
\geometry{hmargin=2.5cm,vmargin=2.5cm}

\title{Derivation of the Black-Scholes Formula}
\author{}
\date{}

\begin{document}

\maketitle

\section{Starting Point: Risk-Neutral Valuation}

The price of a Call option at time $t=0$, denoted by $C_0$, is given by the discounted expectation of the payoff under the risk-neutral measure $\mathbb{Q}$:

\begin{equation}
    C_0 = e^{-rT} \mathbb{E}^{\mathbb{Q}} \left[ \max(S_T - K, 0) \right]
\end{equation}

\noindent Where:
\begin{itemize}
    \item $S_T$ is the asset price at maturity.
    \item $K$ is the strike price.
    \item $r$ is the risk-free rate.
\end{itemize}

\section{Modeling the Underlying Asset}

In the Black-Scholes model, the asset price follows a Geometric Brownian Motion. We can express $S_T$ in terms of a standard normal random variable $Z \sim \mathcal{N}(0,1)$:

\begin{equation}
    S_T = S_0 \exp\left( \left(r - \frac{\sigma^2}{2}\right)T + \sigma\sqrt{T} Z \right)
\end{equation}

\section{Option Exercise Condition}

The option is exercised if $S_T > K$. Let us find the equivalent condition on $Z$:

\begin{align*}
    S_0 \exp\left( \left(r - \frac{\sigma^2}{2}\right)T + \sigma\sqrt{T} Z \right) &> K \\
    \ln(S_0) + \left(r - \frac{\sigma^2}{2}\right)T + \sigma\sqrt{T} Z &> \ln(K) \\
    \sigma\sqrt{T} Z &> \ln\left(\frac{K}{S_0}\right) - \left(r - \frac{\sigma^2}{2}\right)T \\
    Z &> \frac{\ln(K/S_0) - (r - \frac{\sigma^2}{2})T}{\sigma\sqrt{T}}
\end{align*}

To simplify notation, we define the lower bound $-d_2$ such that exercise occurs if $Z > -d_2$. We set:
$$ -d_2 = \frac{\ln(K/S_0) - (r - \frac{\sigma^2}{2})T}{\sigma\sqrt{T}} \quad \Rightarrow \quad d_2 = \frac{\ln(S_0/K) + (r - \frac{\sigma^2}{2})T}{\sigma\sqrt{T}} $$

\section{Calculating the Integral}

We can now rewrite the expectation as an integral using the standard normal density function $\varphi(z) = \frac{1}{\sqrt{2\pi}}e^{-z^2/2}$:

\begin{equation}
    C_0 = e^{-rT} \int_{-d_2}^{+\infty} (S_T(z) - K) \varphi(z) \, dz
\end{equation}

By linearity of the integral, we split the calculation into two terms, $A$ and $B$:

$$ C_0 = \underbrace{e^{-rT} \int_{-d_2}^{+\infty} S_T(z) \varphi(z) \, dz}_{\text{Term A}} - \underbrace{e^{-rT} K \int_{-d_2}^{+\infty} \varphi(z) \, dz}_{\text{Term B}} $$

\subsection{Calculating Term B (Strike)}
This term corresponds to the discounted probability of exercise:
$$ B = K e^{-rT} \int_{-d_2}^{+\infty} \frac{1}{\sqrt{2\pi}} e^{-z^2/2} \, dz $$
The integral represents the probability $\mathbb{P}(Z > -d_2)$. By symmetry of the normal distribution, $\mathbb{P}(Z > -x) = \mathbb{P}(Z < x) = N(x)$.
$$ B = K e^{-rT} N(d_2) $$

\subsection{Calculating Term A (Asset)}
$$ A = e^{-rT} \int_{-d_2}^{+\infty} S_0 \exp\left( \left(r - \frac{\sigma^2}{2}\right)T + \sigma\sqrt{T} z \right) \frac{1}{\sqrt{2\pi}} e^{-z^2/2} \, dz $$
The terms $e^{-rT}$ and $e^{rT}$ cancel out. We factor out $S_0$ and combine the exponentials:
$$ A = S_0 \int_{-d_2}^{+\infty} \frac{1}{\sqrt{2\pi}} \exp\left( -\frac{\sigma^2 T}{2} + \sigma\sqrt{T} z - \frac{z^2}{2} \right) \, dz $$
We complete the square in the exponent:
$$ -\frac{1}{2} (z^2 - 2\sigma\sqrt{T}z + \sigma^2 T) = -\frac{1}{2} (z - \sigma\sqrt{T})^2 $$
The integral becomes:
$$ A = S_0 \int_{-d_2}^{+\infty} \frac{1}{\sqrt{2\pi}} e^{-\frac{1}{2}(z - \sigma\sqrt{T})^2} \, dz $$
Let us perform the change of variable $u = z - \sigma\sqrt{T}$ (so $dz = du$).
The lower bound $-d_2$ becomes $-d_2 - \sigma\sqrt{T}$.
Since we define $d_1 = d_2 + \sigma\sqrt{T}$, the new bound is $-d_1$.

$$ A = S_0 \int_{-d_1}^{+\infty} \frac{1}{\sqrt{2\pi}} e^{-u^2/2} \, du = S_0 N(d_1) $$

\section{Final Result: Black-Scholes Formula}

By combining terms $A$ and $B$, we obtain the Call formula:

\begin{equation}
    \boxed{C_0 = S_0 N(d_1) - K e^{-rT} N(d_2)}
\end{equation}

With:
\begin{align*}
    d_1 &= \frac{\ln(S_0/K) + (r + \frac{\sigma^2}{2})T}{\sigma\sqrt{T}} \\
    d_2 &= d_1 - \sigma\sqrt{T}
\end{align*}

\end{document}