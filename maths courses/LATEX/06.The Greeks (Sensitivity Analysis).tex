\documentclass[a4paper,11pt]{article}

% --- Packages ---
\usepackage[utf8]{inputenc}
\usepackage[T1]{fontenc}
\usepackage[english]{babel}
\usepackage{amsmath, amsfonts, amssymb}
\usepackage{geometry}
\usepackage{parskip}

% --- Layout ---
\geometry{hmargin=2.5cm,vmargin=2.5cm}

% --- Title ---
\title{Step 6: The Greeks\\(Sensitivity Analysis)}
\author{}
\date{}

\begin{document}

\maketitle

\section{Introduction}

The "Greeks" measure the sensitivity of the option price to changes in underlying parameters. They are essential for risk management and hedging.
We start with the Black-Scholes formula for a Call option:
\begin{equation}
    C = S N(d_1) - K e^{-rT} N(d_2)
\end{equation}

\section{Preliminary Mathematical Identity}

Before differentiating, we establish a crucial identity that simplifies calculations.
Recall the definitions of $d_1$ and $d_2$:
$$ d_2 = d_1 - \sigma\sqrt{T} $$
The standard normal density is $\varphi(x) = \frac{1}{\sqrt{2\pi}}e^{-x^2/2}$.

It can be shown that:
\begin{equation} \label{eq:identity}
    S \varphi(d_1) - K e^{-rT} \varphi(d_2) = 0
\end{equation}
\textit{Proof:}
$$ \frac{\varphi(d_1)}{\varphi(d_2)} = \exp\left(-\frac{d_1^2}{2} + \frac{(d_1 - \sigma\sqrt{T})^2}{2}\right) = \exp\left(-\frac{2d_1\sigma\sqrt{T} - \sigma^2 T}{2}\right) = \frac{K e^{-rT}}{S} $$
Rearranging yields the identity.

\section{Derivation of Delta ($\Delta$)}

Delta measures the sensitivity to the stock price $S$:
$$ \Delta = \frac{\partial C}{\partial S} $$

Differentiating the pricing formula using the product rule:
$$ \frac{\partial C}{\partial S} = N(d_1) + S \frac{\partial N(d_1)}{\partial S} - K e^{-rT} \frac{\partial N(d_2)}{\partial S} $$

Since $N'(x) = \varphi(x)$:
$$ \frac{\partial C}{\partial S} = N(d_1) + S \varphi(d_1) \frac{\partial d_1}{\partial S} - K e^{-rT} \varphi(d_2) \frac{\partial d_2}{\partial S} $$

Note that $d_2 = d_1 - \sigma\sqrt{T}$, implying $\frac{\partial d_2}{\partial S} = \frac{\partial d_1}{\partial S}$ (since $\sigma\sqrt{T}$ is constant w.r.t $S$).
We factor out the term $\frac{\partial d_1}{\partial S}$:

$$ \frac{\partial C}{\partial S} = N(d_1) + \frac{\partial d_1}{\partial S} \underbrace{\left[ S \varphi(d_1) - K e^{-rT} \varphi(d_2) \right]}_{\text{Zero by Eq. (\ref{eq:identity})}} $$

Thus, the formula simplifies dramatically to:
\begin{equation}
    \boxed{\Delta_{\text{Call}} = N(d_1)}
\end{equation}
(For a Put option, $\Delta_{\text{Put}} = N(d_1) - 1$).

\section{Derivation of Gamma ($\Gamma$)}

Gamma measures the rate of change of Delta (convexity). It is the same for Calls and Puts.
$$ \Gamma = \frac{\partial \Delta}{\partial S} = \frac{\partial N(d_1)}{\partial S} $$

Applying the chain rule:
$$ \Gamma = \varphi(d_1) \frac{\partial d_1}{\partial S} $$

Recall $d_1 = \frac{\ln(S/K) + (r + \sigma^2/2)T}{\sigma\sqrt{T}}$. The derivative with respect to $S$ is:
$$ \frac{\partial d_1}{\partial S} = \frac{1}{S\sigma\sqrt{T}} $$

Therefore:
\begin{equation}
    \boxed{\Gamma = \frac{\varphi(d_1)}{S\sigma\sqrt{T}}}
\end{equation}


\begin{document}

\maketitle

We rely on the \textbf{Magic Identity} derived previously:
\begin{equation} \label{eq:magic}
    S \varphi(d_1) = K e^{-rT} \varphi(d_2)
\end{equation}

\section{Vega ($\mathcal{V}$): Sensitivity to Volatility}
Vega measures the sensitivity of the option price to changes in volatility $\sigma$.
$$ \mathcal{V} = \frac{\partial C}{\partial \sigma} $$

Differentiating the Black-Scholes formula:
$$ \frac{\partial C}{\partial \sigma} = S \varphi(d_1) \frac{\partial d_1}{\partial \sigma} - K e^{-rT} \varphi(d_2) \frac{\partial d_2}{\partial \sigma} $$

Recall that $d_2 = d_1 - \sigma\sqrt{T}$. Thus:
$$ \frac{\partial d_2}{\partial \sigma} = \frac{\partial d_1}{\partial \sigma} - \sqrt{T} $$

Substituting this back:
$$ \frac{\partial C}{\partial \sigma} = S \varphi(d_1) \frac{\partial d_1}{\partial \sigma} - K e^{-rT} \varphi(d_2) \left( \frac{\partial d_1}{\partial \sigma} - \sqrt{T} \right) $$

Grouping terms by $\frac{\partial d_1}{\partial \sigma}$:
$$ \frac{\partial C}{\partial \sigma} = \frac{\partial d_1}{\partial \sigma} \underbrace{\left( S \varphi(d_1) - K e^{-rT} \varphi(d_2) \right)}_{\text{Zero by Eq. \ref{eq:magic}}} + K e^{-rT} \varphi(d_2) \sqrt{T} $$

Using the Magic Identity on the remaining term ($K e^{-rT} \varphi(d_2) = S \varphi(d_1)$), we get the final formula:

\begin{equation}
    \boxed{\mathcal{V} = S \sqrt{T} \varphi(d_1)}
\end{equation}

\section{Rho ($\rho$): Sensitivity to Interest Rates}
Rho measures sensitivity to the risk-free rate $r$.
$$ \rho = \frac{\partial C}{\partial r} $$

$$ \frac{\partial C}{\partial r} = S \varphi(d_1) \frac{\partial d_1}{\partial r} - K \left[ -T e^{-rT} N(d_2) + e^{-rT} \varphi(d_2) \frac{\partial d_2}{\partial r} \right] $$

Note that $d_1 - d_2 = \sigma\sqrt{T}$, which does not depend on $r$. Therefore $\frac{\partial d_1}{\partial r} = \frac{\partial d_2}{\partial r}$.
The terms involving the partial derivatives cancel out due to the Magic Identity. We are left with the term from differentiating the discount factor:

\begin{equation}
    \boxed{\rho = K T e^{-rT} N(d_2)}
\end{equation}

\section{Theta ($\Theta$): Sensitivity to Time Decay}
Theta measures the change in price as time to maturity $T$ decreases. By convention, Theta is negative.
$$ \Theta = \frac{\partial C}{\partial t} = -\frac{\partial C}{\partial T} $$

Differentiating with respect to $T$:
$$ \frac{\partial C}{\partial T} = S \varphi(d_1) \frac{\partial d_1}{\partial T} - K \left[ -r e^{-rT} N(d_2) + e^{-rT} \varphi(d_2) \frac{\partial d_2}{\partial T} \right] $$

Using $d_2 = d_1 - \sigma\sqrt{T}$, we have $\frac{\partial d_2}{\partial T} = \frac{\partial d_1}{\partial T} - \frac{\sigma}{2\sqrt{T}}$.
Substitute and simplify using the Magic Identity to eliminate $\frac{\partial d_1}{\partial T}$:

$$ \frac{\partial C}{\partial T} = r K e^{-rT} N(d_2) + K e^{-rT} \varphi(d_2) \frac{\sigma}{2\sqrt{T}} $$

Using $K e^{-rT} \varphi(d_2) = S \varphi(d_1)$ again:
$$ \frac{\partial C}{\partial T} = r K e^{-rT} N(d_2) + \frac{S \varphi(d_1) \sigma}{2\sqrt{T}} $$

Since Theta is traditionally the negative of this derivative:

\begin{equation}
    \boxed{\Theta = - \frac{S \varphi(d_1) \sigma}{2\sqrt{T}} - r K e^{-rT} N(d_2)}
\end{equation}

\end{document}
