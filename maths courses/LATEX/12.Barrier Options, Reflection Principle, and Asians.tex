\documentclass[a4paper,11pt]{report}

% =======================================================
% PACKAGES & CONFIGURATION
% =======================================================
\usepackage[utf8]{inputenc}
\usepackage[T1]{fontenc}
\usepackage[english]{babel}
\usepackage{amsmath, amsfonts, amssymb, amsthm}
\usepackage{geometry}
\usepackage{parskip}
\usepackage{fancyhdr}
\usepackage{algorithm}
\usepackage{algpseudocode}
\usepackage{tcolorbox}
\usepackage{hyperref}
\usepackage{booktabs}
\usepackage{tocloft}

% --- Graphics & Plots ---
\usepackage{pgfplots}
\pgfplotsset{compat=1.18}
\usepackage{tikz}
\usetikzlibrary{intersections, backgrounds}

% --- Layout Settings ---
\geometry{hmargin=2.5cm,vmargin=2.5cm}
\hypersetup{colorlinks=true, linkcolor=darkblue, urlcolor=darkblue}
\definecolor{darkblue}{RGB}{0, 0, 100}

% --- Color Definitions ---
\definecolor{exoBlue}{RGB}{0, 60, 120}
\definecolor{exoRed}{RGB}{180, 40, 40}
\definecolor{exoGreen}{RGB}{0, 100, 50}
\definecolor{mirrorGrey}{RGB}{120, 120, 120}

% --- Theorem Styles ---
\newtheorem{theorem}{Theorem}[chapter]
\newtheorem{lemma}{Lemma}[chapter]
\newtheorem{definition}{Definition}[chapter]
\newtheorem{proposition}{Proposition}[chapter]

% --- Header/Footer ---
\pagestyle{fancy}
\fancyhead[L]{\textbf{Stochastic Calculus II}}
\fancyhead[R]{Exotic Options}
\fancyfoot[C]{\thepage}

% =======================================================
% DOCUMENT START
% =======================================================
\begin{document}

% --- TITLE PAGE ---
\begin{titlepage}
    \centering
    \vspace*{2cm}
    {\Huge \textbf{Path-Dependent Derivatives}}\\[0.5cm]
    {\Large \textit{Barrier Options, Reflection Principle, and Asians}}\\[2cm]
    
    \textbf{Lecture 12}\\
    M.Sc. Quantitative Finance\\[2cm]
    
    % Graph: Reflection Principle
    \begin{tikzpicture}
        \begin{axis}[
            width=12cm, height=7cm,
            xlabel={Time ($t$)}, ylabel={Brownian Motion ($W_t$)},
            xmin=0, xmax=10, ymin=-2, ymax=4,
            axis lines=middle,
            title={\textbf{The Reflection Principle}}
        ]
        % Barrier
        \draw[red, thick, dashed] (axis cs:0, 2) -- (axis cs:10, 2);
        \node at (axis cs: 1, 2.2) [red] {Level $b$};
        
        % Original Path (Hitting b and going up)
        \addplot[thick, exoBlue] coordinates {(0,0) (2,1) (4,2) (6,2.5) (8,3) (10,3.5)};
        \node at (axis cs:10, 3.5) [anchor=south] {$W_T$};
        
        % Reflected Path
        \addplot[thick, mirrorGrey, dashed] coordinates {(4,2) (6,1.5) (8,1) (10,0.5)};
        \node at (axis cs:10, 0.5) [anchor=north] {$\tilde{W}_T = 2b - W_T$};
        
        % Intersection point
        \draw[fill=black] (axis cs:4,2) circle (2pt);
        \node at (axis cs:4, 1.8) [anchor=north] {$\tau_b$};
        \end{axis}
    \end{tikzpicture}
    
    \vfill
    {\large \today}
\end{titlepage}

\tableofcontents
\newpage

% =======================================================
% CHAPTER 1: THE REFLECTION PRINCIPLE
% =======================================================
\chapter{Brownian Motion Extremes}

To price Barrier options (which die if $S_t$ hits a level), we need the distribution of the running maximum of a Brownian Motion.

\section{First Hitting Time}
Let $W_t$ be a standard Brownian Motion with $W_0=0$. Let $b > 0$.
We define the \textbf{First Hitting Time} $\tau_b$:
\begin{equation}
    \tau_b = \inf \{ t > 0 : W_t = b \}
\end{equation}

\section{The Reflection Principle}
This is a fundamental result in stochastic processes. It relies on the symmetry of Brownian motion.

\begin{theorem}[Reflection Principle]
For any $b > 0$ and $t > 0$:
\begin{equation}
    P(\tau_b \le t) = 2 P(W_t \ge b) = 2 \left( 1 - N\left(\frac{b}{\sqrt{t}}\right) \right)
\end{equation}
\end{theorem}

\begin{proof}
Consider a path that crosses level $b$ before time $t$ (so $\tau_b \le t$).
At time $T$, this path is either above $b$ ($W_t \ge b$) or below $b$ ($W_t < b$).
\begin{itemize}
    \item Case 1: $W_t \ge b$. This clearly implies $\tau_b \le t$.
    \item Case 2: $W_t < b$ BUT $\tau_b \le t$. This means the path went up to $b$, and then came back down.
\end{itemize}
By the Strong Markov Property, once the Brownian motion hits $b$, it "forgets" the past and restarts. Since Brownian motion is symmetric, for every path that continues UP to $x > b$, there is a mirror path that goes DOWN to $2b - x$.
Therefore:
$$ P(\tau_b \le t, W_t < b) = P(\tau_b \le t, W_t > b) = P(W_t > b) $$
Adding the two cases:
$$ P(\tau_b \le t) = P(W_t > b) + P(W_t > b) = 2 P(W_t > b) $$
\end{proof}

\section{Distribution of the Maximum}
Let $M_t = \max_{0 \le s \le t} W_s$.
The event $\{ M_t \ge b \}$ is identical to $\{ \tau_b \le t \}$.
Thus, the running maximum has the same distribution as $|W_t|$.

% =======================================================
% CHAPTER 2: BARRIER OPTIONS
% =======================================================
\chapter{Barrier Option Pricing}

\section{The Down-and-Out Call}
Consider a Call option with strike $K$ and barrier $B < S_0$.
The option pays $\max(S_T - K, 0)$ if and only if $\min_{0 \le t \le T} S_t > B$.

\section{Derivation Strategy}
We model $S_t$ as a Geometric Brownian Motion:
$$ S_t = S_0 e^{(r - \sigma^2/2)t + \sigma W_t} $$
The condition $S_t > B$ is equivalent to a condition on the arithmetic Brownian motion with drift.
Let $X_t = \ln(S_t/B)$. The barrier for $X_t$ is $0$.
The problem reduces to calculating the probability that a \textbf{Drifted Brownian Motion} stays positive.

\section{The Analytical Formula}
Using the Girsanov theorem to handle the drift in the reflection principle, we obtain the closed-form solution:

\begin{tcolorbox}[colback=blue!5!white,colframe=exoBlue,title=\textbf{Down-and-Out Call Formula}]
Let $\lambda = \frac{r - \sigma^2/2}{\sigma^2}$. Let $y = \ln(B^2 / (S_0 K))$.
\begin{equation}
    C_{DO} = C_{BS}(S_0, K) - \left( \frac{S_0}{B} \right)^{2\lambda + 2} C_{BS}\left( \frac{B^2}{S_0}, K \right)
\end{equation}
\end{tcolorbox}

\textbf{Interpretation:}
\begin{itemize}
    \item The first term is the price of a standard European Call (ignoring the barrier).
    \item The second term is the value of the "Knock-Out" clause. It represents the probability-weighted value of the paths that hit $B$ and would have ended ITM.
    \item $\left( \frac{S_0}{B} \right)^{2\lambda + 2}$ is the discount factor related to the distance from the barrier.
\end{itemize}

% =======================================================
% CHAPTER 3: ASIAN OPTIONS
% =======================================================
\chapter{Asian Options}

\section{Geometric Asian Options (Analytic)}
The payoff depends on the geometric mean $G_T = \exp\left( \frac{1}{T} \int_0^T \ln S_t dt \right)$.

\begin{proposition}[Variance Reduction]
The geometric mean of a GBM is also a Log-Normal variable, but with reduced volatility.
The effective volatility $\sigma_{G}$ is:
\begin{equation}
    \sigma_G = \frac{\sigma}{\sqrt{3}}
\end{equation}
\end{proposition}

\textit{Proof Sketch:}
$\ln S_t = \ln S_0 + \mu t + \sigma W_t$.
The integral $\int_0^T W_t dt$ is a Gaussian variable.
The variance of the average is the average of the covariances.
$$ \text{Var}\left( \frac{1}{T} \int_0^T W_t dt \right) = \frac{1}{T^2} \int_0^T \int_0^T \min(s,t) ds dt = \frac{T}{3} $$
Thus, we can price Geometric Asians using the Black-Scholes formula with $\sigma_{eff} = \sigma/\sqrt{3}$ and adjusted drift $r_{eff} = \frac{1}{2}(r - \sigma^2/6)$.

\section{Arithmetic Asian Options (Numerical)}
The payoff depends on $A_T = \frac{1}{T} \int_0^T S_t dt$.
Since the sum of Log-Normals has no closed-form PDF, we rely on Monte Carlo.

\section{Control Variates Technique}
Standard Monte Carlo is slow. We use the Geometric Asian (whose price $V_G$ is known exactly) to reduce the variance of the Arithmetic estimator.

\begin{algorithm}[H]
\caption{Asian Pricing with Control Variate}
\begin{algorithmic}[1]
\State \textbf{Input:} $S_0, K, r, T, \sigma, M$
\State Calculate exact price $V_{Geom}^{Exact}$ using BS formula with $\sigma/\sqrt{3}$.
\State Initialize sums for $V_{Arith}$ and $V_{Geom}$.
\For{$i = 1$ to $M$}
    \State Simulate path $S_0, \dots, S_T$
    \State Calculate Arithmetic Mean $A_T$ and Payoff $P_A = (A_T - K)^+$
    \State Calculate Geometric Mean $G_T$ and Payoff $P_G = (G_T - K)^+$
    \State Store pair $(P_A, P_G)$
\EndFor
\State Estimate correlation $\beta = \frac{\text{Cov}(P_A, P_G)}{\text{Var}(P_G)}$
\State \textbf{Result:} $\hat{V}_{CV} = \hat{V}_{Arith}^{MC} - \beta (\hat{V}_{Geom}^{MC} - V_{Geom}^{Exact})$
\end{algorithmic}
\end{algorithm}

\begin{center}
\begin{tikzpicture}
    \begin{axis}[
        width=10cm, height=6cm,
        xlabel={Simulations $M$}, ylabel={Standard Error},
        title={Variance Reduction Efficiency},
        set layers,
        ymode=log,
        grid=major
    ]
    % Standard MC Error (1/sqrt(M))
    \addplot[exoRed, thick, domain=100:10000] {1/sqrt(x)};
    \addlegendentry{Standard Monte Carlo}
    
    % Control Variate Error (Much lower)
    \addplot[exoGreen, thick, domain=100:10000] {0.05/sqrt(x)};
    \addlegendentry{With Control Variate}
    \end{axis}
\end{tikzpicture}
\end{center}

\end{document}