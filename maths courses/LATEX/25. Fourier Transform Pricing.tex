\documentclass[a4paper,11pt]{article}

% --- Packages ---
\usepackage[utf8]{inputenc}
\usepackage[T1]{fontenc}
\usepackage[english]{babel}
\usepackage{amsmath, amsfonts, amssymb}
\usepackage{geometry}
\usepackage{parskip}
\usepackage{fancyhdr}

% --- Layout ---
\geometry{hmargin=2.5cm,vmargin=2.5cm}
\pagestyle{fancy}
\fancyhead[L]{Fourier Transform Pricing}
\fancyhead[R]{Quant Finance Portfolio}

\title{\textbf{Step 25: Fourier Transform Pricing}\\ \large The Carr-Madan Formula \& Heston Calibration}
\author{}
\date{}

\begin{document}

\maketitle
\tableofcontents

\section{The Power of Characteristic Functions}
In advanced quantitative finance, many models (Heston, Bates, Variance Gamma) do not have a closed-form Probability Density Function (PDF), but their \textbf{Characteristic Function (CF)} is known analytically.

The characteristic function $\phi(u)$ is the Fourier Transform of the density $f(x)$:
$$ \phi(u) = \int_{-\infty}^{\infty} e^{iux} f(x) dx $$

If we know $\phi(u)$, we can recover the option price without ever knowing the density explicitly.

\section{Case Study: The Heston Model}
The Heston model dynamics are:
\begin{align*}
    dS_t &= \mu S_t dt + \sqrt{v_t} S_t dW_t^S \\
    dv_t &= \kappa (\theta - v_t) dt + \xi \sqrt{v_t} dW_t^v
\end{align*}
with correlation $d \langle W^S, W^v \rangle = \rho dt$.

The characteristic function of the log-price $x_T = \ln(S_T)$ is known (Heston, 1993):
$$ \phi(u) = \exp \left( C(u, \tau) \theta + D(u, \tau) v_0 + i u \ln(S_0) \right) $$
Where $C$ and $D$ are complex-valued functions involving $\sqrt{(\rho \xi u i - \kappa)^2 + \xi^2 (u^2 + ui)}$.

\section{The Carr-Madan Formula (1999)}
Carr and Madan revolutionized calibration by linking the Call Price $C(K)$ directly to $\phi(u)$ via FFT.

They define a **Dampened Call Price**:
$$ c(k) = e^{\alpha k} \times \text{Call}(e^k) $$
Using Parseval's identity or inverse Fourier transforms, they derived:
\begin{equation}
    C(k) = \frac{e^{-\alpha k}}{\pi} \int_0^\infty e^{-ivk} \psi(v) dv
\end{equation}
Where $\psi(v)$ is related to the characteristic function of the log-price:
$$ \psi(v) = \frac{e^{-rT} \phi(v - (\alpha + 1)i)}{\alpha^2 + \alpha - v^2 + i(2\alpha + 1)v} $$

\section{The FFT Implementation}
The goal is to compute the integral for \textbf{many strikes $k$} simultaneously. We approximate the integral as a sum:
$$ I(k_u) \approx \sum_{j=0}^{N-1} e^{-i \frac{2\pi}{N} j u} \psi(v_j) \Delta v $$
This is exactly the format of a Discrete Fourier Transform (DFT).

\subsection{Algorithm Steps}
\begin{enumerate}
    \item \textbf{Discretize} the integration domain $[0, A]$ into $N=4096$ points with spacing $\eta$.
    \item \textbf{Compute} the vector of $\psi(v_j)$ values using the Heston analytic simplifications.
    \item \textbf{Apply FFT}: Use `fft(vector)` to compute the summation in $O(N \log N)$ time.
    \item \textbf{Multiply} by dampening factors and weights (Simpson's rule weights).
    \item \textbf{Extract} Call prices for the grid of Log-Strikes within the strike range.
\end{enumerate}

\section{Why FFT? (Speed Analysis)}
If we need to calibrate the 5 Heston parameters $(\kappa, \theta, \xi, \rho, v_0)$ to a surface of 100 options:
\begin{itemize}
    \item **Direct Integration**: $100 \text{ integrals} \times 1000 \text{ steps} = 10^5$ evaluations.
    \item **FFT**: $1 \text{ FFT call} = 100$ prices instantly.
\end{itemize}
This speedup factor (often 100x or 1000x) is why FFT is the standard for real-time volatility surface calibration.

\end{document}
