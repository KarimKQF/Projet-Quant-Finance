\documentclass[a4paper,11pt]{report}

% =======================================================
% PACKAGES & CONFIGURATION
% =======================================================
\usepackage[utf8]{inputenc}
\usepackage[T1]{fontenc}
\usepackage[english]{babel}
\usepackage{amsmath, amsfonts, amssymb, amsthm}
\usepackage{geometry}
\usepackage{parskip}
\usepackage{fancyhdr}
\usepackage{tcolorbox}
\usepackage{hyperref}
\usepackage{booktabs}

% --- Graphics & Plots ---
\usepackage{pgfplots}
\pgfplotsset{compat=1.18}
\usepackage{tikz}
\usepgfplotslibrary{fillbetween}

% --- Layout Settings ---
\geometry{hmargin=2.5cm,vmargin=2.5cm}
\definecolor{riskRed}{RGB}{180, 20, 20}
\definecolor{safeGreen}{RGB}{0, 100, 50}
\definecolor{tailGrey}{RGB}{100, 100, 100}

% --- Header/Footer ---
\pagestyle{fancy}
\fancyhead[L]{\textbf{Risk Management}}
\fancyhead[R]{VaR \& Expected Shortfall}
\fancyfoot[C]{\thepage}

% --- Theorem Styles ---
\newtheorem{definition}{Definition}[chapter]
\newtheorem{theorem}{Theorem}[chapter]

% =======================================================
% DOCUMENT START
% =======================================================
\begin{document}

% --- TITLE PAGE ---
\begin{titlepage}
    \centering
    \vspace*{2cm}
    {\Huge \textbf{Market Risk Management}}\\[0.5cm]
    {\Large \textit{Value at Risk (VaR) and Expected Shortfall}}\\[2cm]
    
    \textbf{Lecture 13}\\
    M.Sc. Quantitative Finance\\[2cm]
    
    % VaR Visualization Graph
    \begin{tikzpicture}
        \begin{axis}[
            width=12cm, height=7cm,
            xlabel={Loss Amount ($L$)}, ylabel={Probability Density},
            xmin=-2, xmax=6, ymin=0, ymax=0.45,
            axis lines=left,
            title={\textbf{Visualizing Risk Measures (95\%)}}
        ]
        
        % Normal Distribution (Profit/Loss)
        \addplot[name path=dist, thick, black, domain=-2:6, samples=100] 
            {1/(sqrt(2*pi))*exp(-(x-1)^2/2)};
            
        % VaR Line (at x=2.645 for 95% of N(1,1))
        \draw[riskRed, thick, dashed] (axis cs:2.645, 0) -- (axis cs:2.645, 0.4);
        \node at (axis cs:2.7, 0.42) [riskRed, anchor=south] {VaR$_{95\%}$};
        
        % Fill the Tail (Expected Shortfall Area)
        \path[name path=xaxis] (axis cs:2.645,0) -- (axis cs:6,0);
        \addplot[fill=riskRed, opacity=0.3] fill between[of=dist and xaxis, soft clip={domain=2.645:6}];
        
        \node at (axis cs:3.5, 0.05) [anchor=west] {Tail Risk (ES)};
        
        \end{axis}
    \end{tikzpicture}
    
    \vfill
    {\large \today}
\end{titlepage}

\tableofcontents
\newpage

% =======================================================
% CHAPTER 1: VALUE AT RISK (VaR)
% =======================================================
\chapter{Value at Risk (VaR)}

\section{Definition}
Value at Risk is the standard metric for quantifying market risk. It answers the question: \textit{"What is the maximum loss I can expect with $\alpha\%$ confidence over a horizon $T$?"}

Let $L$ be the random variable representing the \textbf{Loss} of the portfolio over time $T$.
(Note: Positive $L$ means we lost money).

\begin{definition}[Value at Risk]
Given a confidence level $\alpha \in (0, 1)$ (typically 95\% or 99\%), the VaR is the smallest number $l$ such that the probability of the loss exceeding $l$ is no larger than $(1-\alpha)$.
\begin{equation}
    \text{VaR}_{\alpha}(L) = \inf \{ l \in \mathbb{R} : P(L > l) \le 1 - \alpha \}
\end{equation}
Mathematically, it is simply the $(1-\alpha)$-quantile of the loss distribution.
\end{definition}

\section{Calculation Methods}

\subsection{1. Parametric VaR (Variance-Covariance)}
Assume returns follow a Normal distribution $R \sim \mathcal{N}(\mu, \sigma^2)$.
$$ \text{VaR}_{\alpha} = \text{Portfolio Value} \times (\sigma \cdot z_{\alpha} - \mu) $$
where $z_{\alpha}$ is the normal quantile (e.g., 1.645 for 95\%, 2.33 for 99\%).
\textit{Pros:} Instant calculation. \textit{Cons:} Assumes normality (underestimates fat tails).

\subsection{2. Historical Simulation}
Take the last 500 days of historical returns. Apply them to today's portfolio. The VaR is simply the 5th percentile worst outcome.
\textit{Pros:} No assumption on distribution. \textit{Cons:} Assumes the past predicts the future.

\subsection{3. Monte Carlo VaR}
Simulate 10,000 paths using Heston or Jump-Diffusion models. Calculate the portfolio value for each. Sort the losses.
\textit{Pros:} Captures non-linearities (Options). \textit{Cons:} Computationally expensive.

% =======================================================
% CHAPTER 2: COHERENT RISK MEASURES
% =======================================================
\chapter{Coherent Risk Measures}

\section{The Failure of VaR}
VaR is not a "perfect" risk measure because it fails the property of \textbf{Sub-additivity}.

\begin{theorem}[Sub-additivity]
A risk measure $\rho$ is sub-additive if:
$$ \rho(X + Y) \le \rho(X) + \rho(Y) $$
Meaning: "Diversification should reduce risk."
\end{theorem}
For certain non-normal distributions, $\text{VaR}(A+B) > \text{VaR}(A) + \text{VaR}(B)$. This implies that merging two portfolios creates \textit{more} risk, which encourages splitting banks into tiny pieces to hide risk.

\section{Expected Shortfall (ES)}
To fix this, Artzner et al. (1999) proposed Expected Shortfall (also called CVaR or TVaR).

\begin{definition}[Expected Shortfall]
ES is the expected loss \textbf{given that} the loss exceeds the VaR.
\begin{equation}
    \text{ES}_{\alpha}(L) = \mathbb{E}[ L \mid L \ge \text{VaR}_{\alpha}(L) ]
\end{equation}
\end{definition}

\begin{tcolorbox}[colback=safeGreen!5!white,title=\textbf{Why Basel III moved to ES}]
VaR tells you \textit{"We are safe 99\% of the time."}
ES tells you \textit{"If the 1\% crisis happens, we will lose \$5 Billion."}
ES captures the "Tail Risk" (the severity of the crash), whereas VaR is blind to anything beyond the threshold.
\end{tcolorbox}

\begin{center}
\begin{tikzpicture}
    \begin{axis}[
        width=10cm, height=6cm,
        xlabel={Time}, ylabel={Risk Estimate},
        title={VaR vs ES (Fat Tailed Distribution)},
        legend pos=north west, grid=major
    ]
    % Normal VaR
    \addplot[blue, thick] coordinates {(0,1) (10,1)};
    \addlegendentry{VaR (99\%)}
    
    % Expected Shortfall (Much higher for fat tails)
    \addplot[red, ultra thick] coordinates {(0,1.8) (10,1.8)};
    \addlegendentry{Exp. Shortfall (99\%)}
    
    \node at (axis cs: 5, 1.4) {Gap = Hidden Risk};
    \end{axis}
\end{tikzpicture}
\end{center}

\end{document}