\documentclass[a4paper,11pt]{report}

% =======================================================
% PACKAGES & CONFIGURATION
% =======================================================
\usepackage[utf8]{inputenc}
\usepackage[T1]{fontenc}
\usepackage[english]{babel}
\usepackage{amsmath, amsfonts, amssymb, amsthm}
\usepackage{geometry}
\usepackage{parskip}
\usepackage{fancyhdr}
\usepackage{algorithm}
\usepackage{algpseudocode}
\usepackage{tcolorbox}
\usepackage{hyperref}
\usepackage{booktabs}
\usepackage{tocloft}

% --- Graphics & Plots ---
\usepackage{pgfplots}
\pgfplotsset{compat=1.18}
\usepackage{tikz}
\usepgfplotslibrary{fillbetween}

% --- Layout Settings ---
\geometry{hmargin=2.5cm,vmargin=2.5cm}
\hypersetup{colorlinks=true, linkcolor=darkblue, urlcolor=darkblue}
\definecolor{darkblue}{RGB}{0, 0, 100}

% --- Color Definitions ---
\definecolor{hestonBlue}{RGB}{0, 60, 120}
\definecolor{hestonRed}{RGB}{180, 40, 40}
\definecolor{hestonGreen}{RGB}{0, 100, 50}

% --- Theorem Styles ---
\newtheorem{theorem}{Theorem}[chapter]
\newtheorem{definition}{Definition}[chapter]
\newtheorem{remark}{Remark}[chapter]

% --- Header/Footer ---
\pagestyle{fancy}
\fancyhead[L]{\textbf{Advanced Stochastic Calculus}}
\fancyhead[R]{The Heston Model}
\fancyfoot[C]{\thepage}

% =======================================================
% DOCUMENT START
% =======================================================
\begin{document}

% --- TITLE PAGE ---
\begin{titlepage}
    \centering
    \vspace*{2cm}
    {\Huge \textbf{The Heston Model}}\\[0.5cm]
    {\Large \textit{Stochastic Volatility Pricing \& Calibration}}\\[2cm]
    
    \textbf{Module:}\\
    M.Sc. Quantitative Finance\\[2cm]
    
    % Decorative Graph: Heston Smile
    \begin{tikzpicture}
        \begin{axis}[
            width=12cm, height=7cm,
            axis lines=left,
            xlabel={Strike Price ($K$)},
            ylabel={Implied Volatility ($\sigma$)},
            xtick=\empty, ytick=\empty,
            domain=0.5:1.5, samples=50
        ]
        \addplot[hestonBlue, ultra thick] {0.2 + 0.8*(x-1)^2 - 0.2*(x-1)};
        \node at (axis cs: 1.0, 0.25) [anchor=south] {The Volatility Skew};
        \end{axis}
    \end{tikzpicture}
    
    \vfill
    {\large \today}
\end{titlepage}

\tableofcontents
\newpage

% =======================================================
% CHAPTER 1: DYNAMICS
% =======================================================
\chapter{The Heston Dynamics}

\section{Motivation: Failing of Black-Scholes}
The Black-Scholes model assumes constant volatility ($\sigma = \text{const}$). This contradicts market reality, where we observe:
\begin{itemize}
    \item \textbf{Volatility Clustering:} Large moves follow large moves.
    \item \textbf{The Leverage Effect:} Volatility tends to increase when the stock price drops.
    \item \textbf{The Smile/Skew:} OTM Puts are more expensive than ATM Calls.
\end{itemize}
Steven Heston (1993) proposed modeling the variance as a stochastic process itself.

\section{The System of SDEs}
Under the Risk-Neutral measure $\mathbb{Q}$, the Heston model is defined by two coupled Stochastic Differential Equations (SDEs):

\begin{tcolorbox}[colback=blue!5!white,colframe=hestonBlue,title=\textbf{Definition: The Heston Model (1993)}]
\begin{align}
    \frac{dS_t}{S_t} &= r dt + \sqrt{v_t} dW_t^S & \text{(Asset Price Process)} \\
    dv_t &= \kappa (\theta - v_t) dt + \xi \sqrt{v_t} dW_t^v & \text{(Variance Process)}
\end{align}
With correlation:
\begin{equation}
    d\langle W^S, W^v \rangle_t = \rho dt
\end{equation}
\end{tcolorbox}

\section{Parameter Interpretation}
\begin{itemize}
    \item $\mathbf{v_t}$: Instantaneous variance.
    \item $\mathbf{\theta}$ (Theta): Long-run average variance.
    \item $\mathbf{\kappa}$ (Kappa): Mean-reversion speed. High $\kappa$ means volatility returns quickly to $\theta$.
    \item $\mathbf{\xi}$ (Xi): The "Vol-of-Vol". Controls the kurtosis of returns (fat tails).
    \item $\mathbf{\rho}$ (Rho): Correlation. Controls the Skew. Typically $\rho \approx -0.7$ for equities.
\end{itemize}

\section{Mathematical Stability: The Feller Condition}
The variance process is a CIR (Cox-Ingersoll-Ross) process. Since variance cannot be negative, we must ensure $v_t$ stays strictly positive.

\begin{theorem}[Feller Condition]
The process $v_t$ remains strictly positive ($v_t > 0$) almost surely if and only if:
\begin{equation}
    2\kappa\theta > \xi^2
\end{equation}
\end{theorem}
\textit{Proof intuition:} The drift $\kappa\theta$ (pushing away from 0) must be strong enough to overcome the diffusion noise $\xi$ near zero.

% =======================================================
% CHAPTER 2: SEMI-CLOSED FORM SOLUTION
% =======================================================
\chapter{Pricing via Fourier Transforms}

The beauty of the Heston model is that despite being complex, it admits a semi-analytical solution. We do not need to solve the PDE directly; we use Characteristic Functions.

\section{The Gil-Pelaez Formula}
By analogy with Black-Scholes ($C = S N(d_1) - K e^{-rT} N(d_2)$), the Heston price is:

\begin{equation}
    C(S, K, v_0, T) = S P_1 - K e^{-rT} P_2
\end{equation}

Here, $P_1$ and $P_2$ are not simple normal CDFs, but probabilities recovered by inverting the characteristic function using the Gil-Pelaez theorem:

\begin{tcolorbox}[colback=red!5!white,colframe=hestonRed,title=\textbf{Integration in the Complex Plane}]
\begin{equation}
    P_j = \frac{1}{2} + \frac{1}{\pi} \int_0^{\infty} \text{Re}\left[ \frac{e^{-i u \ln K} \phi_j(u, v_0, T)}{i u} \right] du
\end{equation}
\end{tcolorbox}

\section{The Characteristic Function $\phi_j$}
The functions $\phi_j$ solve the Affine PDE associated with the Heston model.
Let $\tau = T - t$. The solution is of the form:
$$ \phi_j(u) = \exp(C_j(\tau) + D_j(\tau)v_0 + i u \ln S) $$

The coefficients $C_j$ and $D_j$ are given by (Heston 1993, Albrecher 2007 "Stable Form"):

\begin{align*}
    D_j(\tau) &= \frac{b_j - \rho \xi u i + d_j}{\xi^2} \left( \frac{1 - e^{d_j \tau}}{1 - g_j e^{d_j \tau}} \right) \\
    C_j(\tau) &= \frac{\kappa \theta}{\xi^2} \left( (b_j - \rho \xi u i + d_j) \tau - 2 \ln \left( \frac{1 - g_j e^{d_j \tau}}{1 - g_j} \right) \right)
\end{align*}

Where:
$$ g_j = \frac{b_j - \rho \xi u i + d_j}{b_j - \rho \xi u i - d_j}, \quad d_j = \sqrt{(\rho \xi u i - b_j)^2 - \xi^2 (2 u_j u i - u^2)} $$
With parameters for $P_1$ ($u_1=0.5, b_1 = \kappa - \lambda - \rho \xi$) and $P_2$ ($u_2=-0.5, b_2 = \kappa - \lambda$).

% =======================================================
% CHAPTER 3: SENSITIVITY ANALYSIS
% =======================================================
\chapter{Analysis of the Smile}

This chapter visualizes how Heston parameters shape the Volatility Surface.

\section{Effect of Correlation $\rho$ (The Skew)}
The correlation $\rho$ determines the slope of the smile.

\begin{center}
\begin{tikzpicture}
    \begin{axis}[
        width=10cm, height=6cm,
        xlabel={Strike ($K$)}, ylabel={Implied Vol},
        title={Impact of $\rho$ (Correlation)},
        legend pos=north east, grid=major
    ]
    % Rho = 0
    \addplot[black, dashed, thick, domain=0.8:1.2] {0.2 + 2*(x-1)^2};
    \addlegendentry{$\rho = 0$ (Symmetric)}
    
    % Rho = -0.7
    \addplot[hestonBlue, ultra thick, domain=0.8:1.2] {0.2 + 2*(x-1)^2 - 0.15*(x-1)};
    \addlegendentry{$\rho = -0.7$ (Skewed)}
    \end{axis}
\end{tikzpicture}
\end{center}
\textbf{Conclusion:} A negative correlation (Equity markets) makes OTM Puts expensive, creating a downward slope.

\section{Effect of Vol-of-Vol $\xi$ (The Smile)}
The parameter $\xi$ determines the curvature (convexity).

\begin{center}
\begin{tikzpicture}
    \begin{axis}[
        width=10cm, height=6cm,
        xlabel={Strike ($K$)}, ylabel={Implied Vol},
        title={Impact of $\xi$ (Vol-of-Vol)},
        legend pos=north, grid=major
    ]
    % Low Vol of Vol
    \addplot[hestonGreen, thick, domain=0.8:1.2] {0.2 + 0.5*(x-1)^2};
    \addlegendentry{Low $\xi$ (Flat)}
    
    % High Vol of Vol
    \addplot[hestonRed, ultra thick, domain=0.8:1.2] {0.2 + 4*(x-1)^2};
    \addlegendentry{High $\xi$ (Convex)}
    \end{axis}
\end{tikzpicture}
\end{center}
\textbf{Conclusion:} High $\xi$ increases the value of both OTM Puts and Calls (Fat Tails).

% =======================================================
% CHAPTER 4: SIMULATION & CALIBRATION
% =======================================================
\chapter{Numerical Implementation}

\section{Simulation: Full Truncation Scheme}
Standard Euler discretization fails because $v_t$ can become negative. We use the \textbf{Full Truncation} scheme (Lord et al.):

\begin{algorithm}[H]
\caption{Heston Monte Carlo Path}
\begin{algorithmic}[1]
\State $S_0, v_0, \kappa, \theta, \xi, \rho, T, N$
\State $\Delta t = T/N$
\For{$i=0$ to $N-1$}
    \State Draw $Z_1, Z_2 \sim \mathcal{N}(0,1)$
    \State $Z_S = \rho Z_1 + \sqrt{1-\rho^2} Z_2$ \Comment{Cholesky}
    \State $\tilde{v}_i = \max(v_i, 0)$ \Comment{Truncation for drift}
    \State $v_{i+1} = v_i + \kappa(\theta - \tilde{v}_i)\Delta t + \xi \sqrt{\tilde{v}_i} \sqrt{\Delta t} Z_1$
    \State $S_{i+1} = S_i \exp\left( (r - 0.5\tilde{v}_i)\Delta t + \sqrt{\tilde{v}_i}\sqrt{\Delta t} Z_S \right)$
\EndFor
\end{algorithmic}
\end{algorithm}

\section{Calibration Strategy}
Calibration involves finding $\Theta = \{v_0, \kappa, \theta, \xi, \rho\}$ to minimize the distance between Heston prices and Market prices.

$$ \min_{\Theta} \sum_{i} w_i \left( C_{Heston}(K_i, T_i, \Theta) - C_{Market}(K_i, T_i) \right)^2 $$

This is a non-convex optimization problem, typically solved using **Levenberg-Marquardt** or **Differential Evolution**.

\end{document}