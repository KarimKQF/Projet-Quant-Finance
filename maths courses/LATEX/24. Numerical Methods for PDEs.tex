\documentclass[a4paper,11pt]{article}

% --- Packages ---
\usepackage[utf8]{inputenc}
\usepackage[T1]{fontenc}
\usepackage[english]{babel}
\usepackage{amsmath, amsfonts, amssymb}
\usepackage{geometry}
\usepackage{parskip}
\usepackage{algorithm}
\usepackage{algorithmic}
\usepackage{listings}
\usepackage{fancyhdr}

% --- Layout ---
\geometry{hmargin=2.5cm,vmargin=2.5cm}
\pagestyle{fancy}
\fancyhead[L]{Numerical Methods for PDEs}
\fancyhead[R]{Quant Finance Portfolio}

\title{\textbf{Step 24: Numerical Methods for PDEs}\\ \large Finite Differences, Crank-Nicolson, and Stability Analysis}
\author{}
\date{}

\begin{document}

\maketitle
\tableofcontents

\section{The Need for Numerical Methods}
While Black-Scholes gives a closed-form solution for European options, most exotic derivatives (American options, Barrier options, Bermudans) obey the same PDE but lack analytical solutions due to complex boundaries.

\textbf{The Black-Scholes PDE:}
$$ \frac{\partial V}{\partial t} + \frac{1}{2}\sigma^2 S^2 \frac{\partial^2 V}{\partial S^2} + rS \frac{\partial V}{\partial S} - rV = 0 $$
To solve this numerically, we transform it into a system of algebraic equations.

\section{Discretization}
We truncate the domain: $S \in [0, S_{\max}]$ and $t \in [0, T]$.
\begin{itemize}
    \item Time steps: $\Delta t = T/N$. Points $t_n = n \Delta t$.
    \item Spot steps: $\Delta S = S_{\max}/M$. Points $S_j = j \Delta S$.
\end{itemize}
We denote $V_j^n \approx V(t_n, S_j)$.

\section{Finite Difference Operators}
Using Taylor expansions, we approximate derivatives:
\begin{itemize}
    \item $\frac{\partial V}{\partial t} \approx \frac{V_j^{n+1} - V_j^n}{\Delta t}$ (Forward Difference)
    \item $\frac{\partial V}{\partial S} \approx \frac{V_{j+1}^n - V_{j-1}^n}{2\Delta S}$ (Central Difference)
    \item $\frac{\partial^2 V}{\partial S^2} \approx \frac{V_{j+1}^n - 2V_j^n + V_{j-1}^n}{(\Delta S)^2}$ (Central Second Difference)
\end{itemize}

\section{The Crank-Nicolson Scheme}
The Crank-Nicolson method is the industry standard because it is **unconditionally stable** and has **second-order accuracy** in both time and space: $O(\Delta t^2, \Delta S^2)$.

It takes the average of the Explicit (known) and Implicit (unknown) schemes at $n$ and $n+1$.
$$ \frac{V_j^{n+1} - V_j^n}{\Delta t} = \frac{1}{2} \mathcal{L}(V^{n+1}) + \frac{1}{2} \mathcal{L}(V^n) $$
After grouping terms, we get a linear system:
\begin{equation}
    -\alpha_j V_{j-1}^n + (1 - \beta_j) V_j^n - \gamma_j V_{j+1}^n = \alpha_j V_{j-1}^{n+1} + (1 + \beta_j) V_j^{n+1} + \gamma_j V_{j+1}^{n+1}
\end{equation}
In matrix form: 
$$ \mathbf{A} \mathbf{V}^n = \mathbf{B} \mathbf{V}^{n+1} $$
Where $\mathbf{A}$ and $\mathbf{B}$ are **Tridiagonal Matrices**.

\section{Boundary Conditions}
To solve the system, we need conditions at the edges of the grid ($S=0$ and $S=S_{\max}$).

\subsection{Lower Boundary ($S=0$)}
At $S=0$, the PDE simplifies (as $\sigma S$ and $rS$ terms vanish):
$$ \frac{\partial V}{\partial t} - rV = 0 \implies V_0^n = V_0^{n+1} e^{-r \Delta t} $$

\subsection{Upper Boundary ($S=S_{\max}$)}
For a Call option, as $S \to \infty$, $V \sim S$. So $\frac{\partial^2 V}{\partial S^2} \approx 0$.
We use the linearity condition:
$$ V_M^n = 2V_{M-1}^n - V_{M-2}^n $$
(Dirichlet conditions are also possible: $V_{M} = S_{\max} - Ke^{-r(T-t)}$).

\section{Stability Analysis (Von Neumann)}
Why not use the simpler Explicit method?
Let error evolve as $\epsilon_j^n = \xi^n e^{ik j \Delta S}$.
For the Explicit method, the amplification factor $\xi$ can benefit $|\xi| > 1$ (instability) if:
$$ \Delta t > \frac{(\Delta S)^2}{\sigma^2 S^2} $$
This puts a severe restriction on time steps.
Crank-Nicolson, however, satisfies $|\xi| \leq 1$ for **any** $\Delta t$.

\section{Dealing with American Options}
For American options, we solve the linear system for a temporary candidate $\tilde{V}^n$, then apply the constraint:
$$ V_j^n = \max(\tilde{V}_j^n, \text{Payoff}(S_j)) $$
This is typically solved using the \textbf{PSOR} (Projected Successive Over-Relaxation) algorithm or typically by "operator splitting" (solve then max).

\end{document}
